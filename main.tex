% ==================== CLASE DEL DOCUMENTO ====================
\documentclass[12pt,oneside,letterpaper]{book}

% ==================== PAQUETES BÁSICOS ====================
\usepackage[utf8]{inputenc}           % Codificación de entrada
\usepackage[T1]{fontenc}              % Codificación de fuente
\usepackage[spanish]{babel}           % Traducción de términos automáticos
\usepackage{geometry}                 % Configuración de márgenes
\usepackage{graphicx}                 % Inclusión de imágenes
\usepackage{float}                    % Mejor control de ubicación de figuras/tablas
\usepackage{setspace}                 % Interlineado
\usepackage{fancyhdr}                 % Encabezados y pies de página
\usepackage{titlesec}                 % Formato de títulos de capítulos/secciones
\usepackage{tocloft}                  % Personalización del índice
\usepackage{caption}                  % Estilo de captions
\usepackage{lipsum}                   % Texto de prueba
\usepackage[hidelinks]{hyperref}     % Enlaces sin color
\usepackage{mathptmx}                 % Fuente Times New Roman

% ==================== CONFIGURACIÓN DE CAPTIONS ====================
\captionsetup{
  font=small,
  labelfont=bf,
  justification=centering
}

% ==================== BIBLIOGRAFÍA (APA) ====================
\usepackage{csquotes}                 % Recomendado por biblatex
\usepackage[style=apa, backend=biber, language=spanish]{biblatex}
\addbibresource{bibliography.bib}    % Archivo .bib con referencias

% ==================== MÁRGENES Y ESPACIADO ====================
\geometry{
  letterpaper,
  left=30mm,      % Recomendado si requiere empastado
  right=25.4mm,
  top=25.4mm,
  bottom=25.4mm,
  headsep=10mm
}
\setlength{\headheight}{15pt}        % Altura del encabezado
\setstretch{1.5}                      % Interlineado 1.5 para APA

% ==================== IMÁGENES ====================
\graphicspath{{images/}}             % Carpeta por defecto para imágenes

% ==================== TÍTULOS Y SECCIONES ====================
\renewcommand{\chaptername}{CAPÍTULO}    % Nombre de capítulo en mayúsculas
\renewcommand{\thechapter}{\Roman{chapter}} % Números romanos
\renewcommand{\thesection}{\arabic{chapter}.\arabic{section}} % Secciones tipo 1.1

% Formato de capítulos
\titleformat{\chapter}[block]
  {
\normalfont\bfseries\normalsize}  % Estilo de texto del título
  {CAPÍTULO \thechapter\-}            % Prefijo del capítulo
  {1em}{}

% Formato de secciones
\titleformat{\section}
  {
\normalfont\bfseries\normalsize}
  {\thesection}{1em}{}

% Formato de subsecciones
\titleformat{\subsection}
  {
\normalfont\bfseries\normalsize}
  {\thesubsection}{1em}{}

% Espaciado vertical para capítulos
\titlespacing*{\chapter}{0pt}{-20pt}{30pt} % Ajusta -20pt según se requiera

% Espaciado reducido para título del índice
\usepackage{etoolbox}
\makeatletter
\patchcmd{\tableofcontents}{\@starttoc{toc}}{\vspace*{-1.5cm}\@starttoc{toc}}{}{}
\patchcmd{\listoftables}{\@starttoc{lot}}{\vspace*{-1.5cm}\@starttoc{lot}}{}{}
\patchcmd{\listoffigures}{\@starttoc{lof}}{\vspace*{-1.5cm}\@starttoc{lof}}{}{}
\makeatother

% ==================== ÍNDICE PERSONALIZADO ====================
\renewcommand{\cftchappresnum}{CAPÍTULO }
\renewcommand{\cftchapaftersnum}{\quad}
\renewcommand{\cftchapnumwidth}{8em}

% ==================== ENCABEZADO APA ====================
\pagestyle{fancy}
\fancyhf{}
\fancyhead[R]{\thepage}              % Número de página a la derecha arriba
\renewcommand{\headrulewidth}{0pt}
\renewcommand{\footrulewidth}{0pt}

% Aplica fancy en lugar de plain (índice, capítulos, etc.)
\makeatletter
\let\ps@plain\ps@fancy
\makeatother

% ================= DOCUMENTO =================
\begin{document}
\pagenumbering{gobble}

% ============ PORTADA =============
\thispagestyle{empty}
\begingroup
\setstretch{1.15}
\begin{center}
	\Large\textbf{UNIVERSIDAD MAYOR DE SAN ANDRÉS} \\
    \vspace{0.5cm}
	\large\textbf{FACULTAD DE CIENCIAS PURAS Y NATURALES} \\
	\large\textbf{POSTGRADO EN INFORMÁTICA} \\
	\vspace{0.5cm}
	\large\textbf{DIPLOMADO EN ANÁLISIS DE DATOS Y BUSINESS INTELLIGENCE} \\
	\vspace{0.5cm}
	\large\textbf{MODALIDAD VIRTUAL} \\
	\large\textbf{GESTIÓN 2024 - 2025} \\
	\vspace{0.5cm}
	\includegraphics[width=4cm]{logo_umsa.png}
	\vspace{0.5cm}
	
	\large\textbf{MONOGRAFÍA DE GRADO} \\
	\vspace{0.2cm}
	\large\textbf{MI TÍTULO \\
			SEGUNDA LINEA TITULO} \\
	\vspace{0.5cm}
	\textbf{POR:} LIC. TU NOMBRE \\
	\textbf{TUTOR:} M. SC. MARCELO PALMA SALAS \\
	\vspace{0.5cm}
	LA PAZ – BOLIVIA \\
	Mayo, 2025
\end{center}
\endgroup

% ============ ÍNDICE =============
\newpage
\pagenumbering{roman}
\vspace*{-2cm} % ← Esto reduce el espacio antes del título
\renewcommand{\contentsname}{ÍNDICE}
\renewcommand{\cfttoctitlefont}{\hfill\normalsize\bfseries}
\renewcommand{\cftaftertoctitle}{\hfill}
\tableofcontents

% ============ ÍNDICE DE TABLAS =============
\newpage
\vspace*{-2cm}
\renewcommand{\listtablename}{ÍNDICE DE TABLAS}
\renewcommand{\cftlottitlefont}{\hfill\normalsize\bfseries}
\renewcommand{\cftafterlottitle}{\hfill}
\renewcommand{\cfttabdotsep}{\cftdotsep}
\listoftables

% ============ ÍNDICE DE FIGURAS =============
\newpage
\vspace*{-2cm}
\renewcommand{\listfigurename}{ÍNDICE DE FIGURAS}
\renewcommand{\cftloftitlefont}{\hfill\normalsize\bfseries}
\renewcommand{\cftafterloftitle}{\hfill}
\renewcommand{\cftfigdotsep}{\cftdotsep}
\listoffigures

% ============ RESUMEN Y ABSTRACT ============
\newpage
\section*{Resumen}
\lipsum[1]

\bigskip
\noindent \textbf{Palabras clave:} Evaluación docente, aprendizaje supervisado, clusterización, inteligencia artificial.

\newpage
\section*{Abstract}
\lipsum[1]

\bigskip
\noindent \textbf{Keywords:} Teaching evaluation, supervised learning, clustering, artificial intelligence.

% ============ CAPÍTULO I: MARCO INTRODUCTORIO =============
\newpage
\pagenumbering{arabic}
\thispagestyle{empty}
\vspace*{0.35\textheight}
\begin{center}
	{\Huge\textbf{CAPÍTULO I}} \\[0.5cm]
	{\Huge\textbf{MARCO INTRODUCTORIO}}
\end{center}
\newpage
\thispagestyle{fancy}

% Si quieres que "INTRODUCCIÓN" esté centrado y en el índice:
\section*{}
\begin{center}
	\normalsize \textbf{INTRODUCCIÓN}
\end{center}
\addcontentsline{toc}{section}{Introducción}

\chapter{MARCO INTRODUCTORIO}
\thispagestyle{fancy}

% ============ SECCIONES Y SUBSECCIONES ============
\section{Planteamiento del Problema}
La educación, como pilar fundamental del desarrollo humano, ha experimentado transformaciones en las últimas décadas...

\section{Antecedentes del problema}
\lipsum[1]

\section{Problema de Investigación}
\lipsum[1]

\section{Formulación del Problema}
¿EJEMPLO DE PREGUNTA De qué manera los docentes pueden integrar Inteligencia Artificial Generativa y Realidad Aumentada en las aulas de Humanidades para mejorar la calidad del aprendizaje estudiantil?

\section{Justificación}
\lipsum[1]

\subsection{Justificación Teórica}
\lipsum[1]

\subsection{Justificación Práctica}
\lipsum[1]

\subsection{Pertinencia Social}
\lipsum[1]

\section{Objeto de Estudio}
\lipsum[1]

\section{Objetivos}
\lipsum[1]

\subsection{Objetivo General}
\lipsum[1]

\subsection{Objetivos Específicos}
\lipsum[1]

\section{Alcances y Límites}
\lipsum[1]

\subsection{Alcances}
\lipsum[1]

\subsection{Límites}
\lipsum[1]

% ============ SIGUIENTES CAPÍTULOS (estructura igual) ============
\newpage
\thispagestyle{empty}
\vspace*{0.35\textheight}
\begin{center}
	{\Huge\textbf{CAPÍTULO II}} \\[0.5cm]
	{\Huge\textbf{MARCO TEÓRICO}}
\end{center}

\newpage
\chapter{MARCO TEÓRICO}
\thispagestyle{fancy}
\section{teórico 1}
\lipsum[1]
Ejemplo de cita con autor en el texto: \textcite{perez2021educacion}.\\
Ejemplo de cita entre paréntesis: \parencite{smith2020ai}.

\section{teórico 2}
\lipsum[1]

\section{Ejemplo de Imagen}
\begin{figure}[H]
\centering
\includegraphics[width=0.5\textwidth]{images/logo_umsa.png} % Reemplaza con el nombre de tu imagen
\caption{Ejemplo de imagen ilustrativa}
\label{fig:ejemplo1}
\end{figure}

\newpage
\thispagestyle{empty}
\vspace*{0.35\textheight}
\begin{center}
	{\Huge\textbf{CAPÍTULO III}} \\[0.5cm]
	{\Huge\textbf{MARCO APLICATIVO}}
\end{center}

\newpage
\chapter{MARCO APLICATIVO}
\thispagestyle{fancy}

\section{aplicativo 1}
\lipsum[1]
\section{Ejemplo de Tabla}

\begin{table}[H]
\centering
\begin{tabular}{|l|c|r|}
\hline
Nombre & Edad & Ciudad \\
\hline
Ana & 25 & La Paz \\
Carlos & 30 & Cochabamba \\
Luisa & 28 & Santa Cruz \\
\hline
\end{tabular}
\caption{Ejemplo de tabla con datos ficticios}
\label{tab:ejemplo1}
\end{table}

\newpage
\thispagestyle{empty}
\vspace*{0.35\textheight}
\begin{center}
	{\Huge\textbf{CAPÍTULO IV}} \\[0.5cm]
	{\Huge\textbf{CONCLUSIONES Y RECOMENDACIONES}}
\end{center}

\newpage
\chapter{CONCLUSIONES Y RECOMENDACIONES}
\thispagestyle{fancy}

\section{Conclusiones}
\lipsum[1]

\section{Recomendaciones}
\lipsum[1]

% ============ BIBLIOGRAFÍA ============
\printbibliography[
	heading=bibintoc,
	title={BIBLIOGRAFÍA}
]

\end{document}
