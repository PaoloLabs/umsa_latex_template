\documentclass[12pt,oneside,letterpaper]{book}

% ================= PREÁMBULO =================
\usepackage[utf8]{inputenc}
\usepackage[T1]{fontenc}
\usepackage[spanish]{babel}
\usepackage{geometry}
\usepackage{graphicx}
\usepackage{fancyhdr}
\usepackage{setspace}
\usepackage{float}
\usepackage{titlesec}
\usepackage{tocloft}    % Para personalizar el índice
\usepackage{lipsum}
\usepackage[hidelinks]{hyperref}  % Índice con links ocultos para el PDF final
\usepackage{mathptmx} % Usa Times New Roman como fuente principal
\usepackage{caption} % <--- Aquí
\captionsetup{font=small, labelfont=bf, justification=centering}

% Bibliografía
\usepackage{csquotes} % Recomendado por biblatex
\usepackage[style=apa, backend=biber, language=spanish]{biblatex}
\addbibresource{bibliography.bib}

% Márgenes y espaciado según APA (ajusta left=30mm si tu universidad lo exige)
\geometry{
    letterpaper,
    left=30mm, % Recomendado si es para empastado físico (consulta si es necesario)
    right=25.4mm,
    top=25.4mm,
    bottom=25.4mm,
    headsep=10mm
}
\setlength{\headheight}{15pt}
\setstretch{1.5} % Usa 2 si exigen doble espacio exacto

% Ruta para imágenes
\graphicspath{{images/}}

% Mostrar "CAPÍTULO" en mayúsculas y números romanos para capítulos
\renewcommand{\chaptername}{CAPÍTULO}
\renewcommand{\thechapter}{\Roman{chapter}}

% Secciones numeradas como 1.1, 1.2, etc.
\renewcommand{\thesection}{\arabic{chapter}.\arabic{section}}

% Formato visual del título del capítulo
\titleformat{\chapter}[block]
  {\normalfont\bfseries\large}
  {CAPÍTULO \thechapter\-}
  {1em}
  {}

  \usepackage{titlesec} % ya lo tienes

\titleformat{\section}
  {\normalfont\bfseries\large}   % ← Estilo de la sección: tamaño large y en negrita
  {\thesection}                  % ← Muestra el número (por ejemplo 1.1)
  {1em}                          % ← Espacio entre el número y el título
  {}                             % ← Antes del título, puedes poner algo si quieres

\titleformat{\subsection}
  {\normalfont\bfseries\large}
  {\thesubsection}
  {1em}
  {}


% ========= Personalización del índice =========
\renewcommand{\cftchappresnum}{CAPÍTULO }
\renewcommand{\cftchapaftersnum}{\quad}
\renewcommand{\cftchapnumwidth}{8em}

% ========= Encabezado APA =========
\pagestyle{fancy}
\fancyhf{}
\fancyhead[R]{\thepage}
\renewcommand{\headrulewidth}{0pt}
\renewcommand{\footrulewidth}{0pt}

% Espaciado visual del título de capítulo
\titlespacing*{\chapter}{0pt}{-30pt}{30pt} % Ajusta -10pt según gusto visual

% ================= DOCUMENTO =================
\begin{document}
\pagenumbering{gobble}

% ============ PORTADA =============
\thispagestyle{empty}
\begingroup
\setstretch{1.15}
\begin{center}
	\Large\textbf{UNIVERSIDAD MAYOR DE SAN ANDRÉS} \\
    \vspace{0.5cm}
	\large\textbf{FACULTAD DE CIENCIAS PURAS Y NATURALES} \\
	\large\textbf{POSTGRADO EN INFORMÁTICA} \\
	\vspace{0.5cm}
	\large\textbf{DIPLOMADO EN ANÁLISIS DE DATOS Y BUSINESS INTELLIGENCE} \\
	\vspace{0.5cm}
	\large\textbf{MODALIDAD VIRTUAL} \\
	\large\textbf{GESTIÓN 2024 - 2025} \\
	\vspace{0.5cm}
	\includegraphics[width=4cm]{logo_umsa.png}
	\vspace{0.5cm}
	
	\large\textbf{MONOGRAFÍA DE GRADO} \\
	\vspace{0.2cm}
	\large\textbf{MI TÍTULO \\
			SEGUNDA LINEA TITULO} \\
	\vspace{0.5cm}
	\textbf{POR:} LIC. TU NOMBRE \\
	\textbf{TUTOR:} M. SC. MARCELO PALMA SALAS \\
	\vspace{0.5cm}
	LA PAZ – BOLIVIA \\
	Mayo, 2025
\end{center}
\endgroup

% ============ ÍNDICE =============
\newpage
\thispagestyle{empty}
\renewcommand{\cftchapdotsep}{\cftdotsep}
\renewcommand{\contentsname}{ÍNDICE}
\renewcommand{\cfttoctitlefont}{\large\bfseries}
\tableofcontents

% ============ ÍNDICE DE TABLAS =============
\newpage
\thispagestyle{empty}
\renewcommand{\listtablename}{ÍNDICE DE TABLAS}
\renewcommand{\cftlottitlefont}{\large\bfseries}
\renewcommand{\cfttabdotsep}{\cftdotsep}
\listoftables

% ============ ÍNDICE DE IMÁGENES =============
\newpage
\thispagestyle{empty}
\renewcommand{\listfigurename}{ÍNDICE DE IMÁGENES}
\renewcommand{\cftloftitlefont}{\large\bfseries}
\renewcommand{\cftfigdotsep}{\cftdotsep}
\listoffigures

% ============ RESUMEN Y ABSTRACT ============
\newpage
\pagenumbering{roman} % ← Numeración romana (i, ii, iii)
\section*{Resumen}
\lipsum[1]

\bigskip
\noindent \textbf{Palabras clave:} Evaluación docente, aprendizaje supervisado, clusterización, inteligencia artificial.

\newpage
\section*{Abstract}
\lipsum[1]

\bigskip
\noindent \textbf{Keywords:} Teaching evaluation, supervised learning, clustering, artificial intelligence.

% ============ CAPÍTULO I: MARCO INTRODUCTORIO =============
\newpage
\pagenumbering{arabic}
\thispagestyle{empty}
\vspace*{0.35\textheight}
\begin{center}
	{\Huge\textbf{CAPÍTULO I}} \\[0.5cm]
	{\Huge\textbf{MARCO INTRODUCTORIO}}
\end{center}
\newpage
\thispagestyle{fancy}

% Si quieres que "INTRODUCCIÓN" esté centrado y en el índice:
\section*{}
\begin{center}
	\large \textbf{INTRODUCCIÓN}
\end{center}
\addcontentsline{toc}{section}{Introducción}

\chapter{MARCO INTRODUCTORIO}
\thispagestyle{fancy}

% ============ SECCIONES Y SUBSECCIONES ============
\section{Planteamiento del Problema}
La educación, como pilar fundamental del desarrollo humano, ha experimentado transformaciones en las últimas décadas...

\section{Antecedentes del problema}
\lipsum[1]

\section{Problema de Investigación}
\lipsum[1]

\section{Formulación del Problema}
¿EJEMPLO DE PREGUNTA De qué manera los docentes pueden integrar Inteligencia Artificial Generativa y Realidad Aumentada en las aulas de Humanidades para mejorar la calidad del aprendizaje estudiantil?

\section{Justificación}
\lipsum[1]

\subsection{Justificación Teórica}
\lipsum[1]

\subsection{Justificación Práctica}
\lipsum[1]

\subsection{Pertinencia Social}
\lipsum[1]

\section{Objeto de Estudio}
\lipsum[1]

\section{Objetivos}
\lipsum[1]

\subsection{Objetivo General}
\lipsum[1]

\subsection{Objetivos Específicos}
\lipsum[1]

\section{Alcances y Límites}
\lipsum[1]

\subsection{Alcances}
\lipsum[1]

\subsection{Límites}
\lipsum[1]

% ============ SIGUIENTES CAPÍTULOS (estructura igual) ============
\newpage
\thispagestyle{empty}
\vspace*{0.35\textheight}
\begin{center}
	{\Huge\textbf{CAPÍTULO II}} \\[0.5cm]
	{\Huge\textbf{MARCO TEÓRICO}}
\end{center}

\newpage
\chapter{MARCO TEÓRICO}
\thispagestyle{fancy}
\section{teórico 1}
\lipsum[1]
Ejemplo de cita con autor en el texto: \textcite{perez2021educacion}.\\
Ejemplo de cita entre paréntesis: \parencite{smith2020ai}.

\section{teórico 2}
\lipsum[1]

\section{Ejemplo de Imagen}
\begin{figure}[H]
\centering
\includegraphics[width=0.5\textwidth]{images/logo_umsa.png} % Reemplaza con el nombre de tu imagen
\caption{Ejemplo de imagen ilustrativa}
\label{fig:ejemplo1}
\end{figure}

\newpage
\thispagestyle{empty}
\vspace*{0.35\textheight}
\begin{center}
	{\Huge\textbf{CAPÍTULO III}} \\[0.5cm]
	{\Huge\textbf{MARCO APLICATIVO}}
\end{center}

\newpage
\chapter{MARCO APLICATIVO}
\thispagestyle{fancy}

\section{aplicativo 1}
\lipsum[1]
\section{Ejemplo de Tabla}

\begin{table}[H]
\centering
\begin{tabular}{|l|c|r|}
\hline
Nombre & Edad & Ciudad \\
\hline
Ana & 25 & La Paz \\
Carlos & 30 & Cochabamba \\
Luisa & 28 & Santa Cruz \\
\hline
\end{tabular}
\caption{Ejemplo de tabla con datos ficticios}
\label{tab:ejemplo1}
\end{table}

\newpage
\thispagestyle{empty}
\vspace*{0.35\textheight}
\begin{center}
	{\Huge\textbf{CAPÍTULO IV}} \\[0.5cm]
	{\Huge\textbf{CONCLUSIONES Y RECOMENDACIONES}}
\end{center}

\newpage
\chapter{CONCLUSIONES Y RECOMENDACIONES}
\thispagestyle{fancy}

\section{Conclusiones}
\lipsum[1]

\section{Recomendaciones}
\lipsum[1]

% ============ BIBLIOGRAFÍA ============
\printbibliography[
	heading=bibintoc,
	title={BIBLIOGRAFÍA}
]

\end{document}
