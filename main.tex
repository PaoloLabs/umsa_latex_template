\documentclass[11pt,oneside,letterpaper]{book}

% ================= PREÁMBULO =================
\usepackage[utf8]{inputenc}
\usepackage[T1]{fontenc}
\usepackage[spanish]{babel}
\usepackage{geometry}
\usepackage{graphicx}
\usepackage{fancyhdr}
\usepackage{setspace}
\usepackage{titlesec}
\usepackage{tocloft}    % Para personalizar el índice
\usepackage{lipsum}
%\usepackage{hyperref} % indice con links visibles para latex (descomenta para debug) %
\usepackage[hidelinks]{hyperref}  % indice con links ocultos para convertir a PDF final %
\usepackage{mathptmx} % Usa Times New Roman como fuente principal
\usepackage{graphicx}

% Bibliografía
\usepackage{csquotes} % Recomendado por biblatex
\usepackage[style=apa, backend=biber, language=spanish]{biblatex}
\addbibresource{bibliography.bib}

% Márgenes y espaciado
\geometry{letterpaper, left=40mm, right=25mm, top=30mm, bottom=25mm, headsep=20mm}
\setstretch{1.5}

% Ruta para imagenes
\graphicspath{{images/}}

% Mostrar "CAPÍTULO" en mayúsculas
\renewcommand{\chaptername}{CAPÍTULO}

% Usar números romanos para los capítulos
\renewcommand{\thechapter}{\Roman{chapter}}

% Secciones en formato arábigo (e.g., 4.1)
\renewcommand{\thesection}{\arabic{chapter}.\arabic{section}}

% Formato visual del título del capítulo en el documento
\titleformat{\chapter}[block]
  {\normalfont\bfseries\Large}
  {CAPÍTULO \thechapter\-}{1em}{}

% ========= Personalización del índice =========

% Poner "CAPÍTULO I -" como parte de la numeración
\renewcommand{\cftchappresnum}{CAPÍTULO }  % Usamos \Roman{chapter} aquí también

% Quita el número automático (ya lo incluimos en el presnum)
\renewcommand{\cftchapaftersnum}{\quad}

% Define espacio suficiente para el texto "CAPÍTULO I -"
\renewcommand{\cftchapnumwidth}{8em}

% Cabecera
\pagestyle{fancy}
\fancyhf{}
% \fancyhead[R]{\thepage} % Número de página en la esquina superior derecha
\rfoot{\thepage}
\renewcommand{\headrulewidth}{0pt} % Quita línea del encabezado
\renewcommand{\footrulewidth}{0pt} % Quita línea del pie

\titlespacing*{\chapter}{0pt}{-20pt}{30pt}

% ================= DOCUMENTO =================
\begin{document}
        \pagenumbering{gobble} % quita numeracion
	% CARÁTULA
	\thispagestyle{empty}
	\begin{center}
		\Large{\textbf{UNIVERSIDAD MAYOR DE SAN ANDRÉS}} \\
		\large{\textbf{FACULTAD DE CIENCIAS PURAS Y NATURALES}} \\
		\large{\textbf{POSTGRADO EN INFORMÁTICA}} \\
		\vspace{0.5cm}
		\large{\textbf{DIPLOMADO EN ANÁLISIS DE DATOS Y BUSINESS INTELLIGENCE}} \\
		\vspace{0.5cm}
            \large{\textbf{MODALIDAD VIRTUAL}} \\
		\large{\textbf{GESTIÓN 2024 - 2025}} \\
		\begin{center}
			\includegraphics[width=4cm]{logo_umsa.png}
		\end{center}
		\vspace{0.5cm}
		\large{\textbf{MONOGRAFÍA DE GRADO}} \\
		\vspace{0.2cm}
		\large{\textbf{MI TITULO \\
            SEGUNDA LINEA TITULO}} \\
		\vspace{0.5cm}
		\textbf{POR:} LIC. TU NOMBRE \\
		\textbf{TUTOR:} M. SC. MARCELO PALMA SALAS \\
		\vspace{0.5cm}
		LA PAZ – BOLIVIA \\
		Mayo, 2025
	\end{center}

	\newpage
	\thispagestyle{empty}
        \renewcommand{\cftchapdotsep}{\cftdotsep} % activa puntos suspensivos para capítulos
	   	\renewcommand{\contentsname}{ÍNDICE}
        % Cambia el tamaño del título del índice
        \renewcommand{\cfttoctitlefont}{\large\bfseries}  % o \Large, \normalsize, etc.
        \tableofcontents

	\newpage
        \pagenumbering{roman} % ← Numeración romana (i, ii, iii)
	\vspace*{\fill}
            \begin{center}
	            \textbf{Resumen}
            \end{center}
	    	\lipsum[1]

			\vspace{1em}
			\textbf{Palabras clave:} Tecnologías emergentes, inteligencia artificial generativa, realidad aumentada, competencias digitales, educación superior.

        \vspace*{\fill}

	\newpage
        \vspace*{\fill}
            \begin{center}
	            \textbf{Abstract}
            \end{center}
    		\lipsum[1]

			\vspace{1em}
			\textbf{Keywords:} Emerging technologies, generative artificial intelligence, augmented reality, digital skills, higher education.

        \vspace*{\fill}

    \newpage
    \pagenumbering{arabic}
	\chapter{MARCO INTRODUCTORIO}
	La educación, como pilar fundamental del desarrollo humano, ha experimentado transformaciones en las últimas décadas... [introducción truncada para brevedad]
        \section{Antecedentes del problema}
        \lipsum[1]

        \section{Problema de Investigación}
        \lipsum[1]

        \section{Formulación del Problema}
        ¿EJEMPLO DE PREGUNTA De qué manera los docentes pueden integrar Inteligencia Artificial Generativa y Realidad Aumentada en las aulas de Humanidades para mejorar la calidad del aprendizaje estudiantil?

        \section{Justificación}
        \lipsum[1]

            \subsection{Justificación Teórica}
            \lipsum[1]

            \subsection{Justificación Práctica}
            \lipsum[1]

            \subsection{Pertinencia Social}
            \lipsum[1]

        \section{Objeto de Estudio}
        \lipsum[1]

        \section{Objetivos}
        \lipsum[1]

            \subsection{Objetivo General}
            \lipsum[1]

            \subsection{Objetivos Específicos}
            \lipsum[1]

        \section{Alcances y Límites}
        \lipsum[1]

            \subsection{Alcances}
            \lipsum[1]

            \subsection{Límites}
            \lipsum[1]

	\chapter{MARCO TEÓRICO}
	\lipsum[1-2]
	Ejemplo de cita con autor en el texto: \textcite{perez2021educacion}.\\
	Ejemplo de cita entre paréntesis: \parencite{smith2020ai}.
        \section{teórico 1}
        \lipsum[1]

        \section{teórico 2}
        \lipsum[1]

	\chapter{MARCO APLICATIVO}
	\lipsum[1]

        \section{aplicativo 1}
        \lipsum[1]

        \section{aplicativo 2}
        \lipsum[1]

	\chapter{CONCLUSIONES Y RECOMENDACIONES}

        \section{Conclusiones}
        \lipsum[1]

        \section{Recomendaciones}
        \lipsum[1]

        \printbibliography[
          heading=bibintoc,         % Agrega al índice
          title={BIBLIOGRAFÍA}      % Cambia el título
        ]

\end{document}
