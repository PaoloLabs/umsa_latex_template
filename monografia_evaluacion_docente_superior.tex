\documentclass[11pt,oneside,letterpaper]{book}

% ================= PREÁMBULO =================
\usepackage[utf8]{inputenc}
\usepackage[T1]{fontenc}
\usepackage[spanish]{babel}
\usepackage{geometry}
\usepackage{graphicx}
\usepackage{fancyhdr}
\usepackage{setspace}
\usepackage{titlesec}
\usepackage{tocloft}
\usepackage{lipsum}
\usepackage{mathptmx} % Usa Times New Roman como fuente principal
\usepackage{graphicx}

% Bibliografía
\usepackage{csquotes} % Recomendado por biblatex
\usepackage[style=apa, backend=biber, language=spanish]{biblatex}
\addbibresource{referencias.bib}

% Márgenes y espaciado
\geometry{letterpaper, left=40mm, right=25mm, top=30mm, bottom=25mm, headsep=20mm}
\setstretch{1.5}

% Ruta para imagenes
\graphicspath{{images/}}

% Cabecera
\pagestyle{fancy}
\fancyhf{}
\rfoot{\thepage}

% ================= DOCUMENTO =================
\begin{document}

	% CARÁTULA
	\thispagestyle{empty}
	\begin{center}
		\Large{\textbf{UNIVERSIDAD MAYOR DE SAN ANDRÉS}} \\
		\large{\textbf{FACULTAD DE CIENCIAS PURAS Y NATURALES}} \\
		\large{\textbf{POSTGRADO EN INFORMÁTICA}} \\
		\vspace{0.5cm}
		\large{\textbf{DIPLOMADO EN EDUCACIÓN SUPERIOR CON APLICACIÓN DE TECNOLOGÍAS EMERGENTES E INTELIGENCIA ARTIFICIAL}} \\
		\large{\textbf{MODALIDAD VIRTUAL}} \\
		\large{\textbf{GESTIÓN 2024 - 2025}} \\
		\begin{center}
			\includegraphics[width=4cm]{logo_umsa.png}
		\end{center}
		\vspace{0.5cm}
		\large{\textbf{MONOGRAFÍA DE GRADO}} \\
		\large{\textbf{TECNOLOGÍAS EMERGENTES EN INTELIGENCIA ARTIFICIAL GENERATIVA Y REALIDAD AUMENTADA}} \\
		\large{\textbf{PARA EL DESARROLLO DE COMPETENCIAS DIGITALES}} \\
		\large{\textbf{EN LA FACULTAD DE HUMANIDADES DE LA UNIVERSIDAD MAYOR DE SAN SIMÓN}} \\
		\vspace{0.5cm}
		\textbf{POR:} LIC. CHRISTIAN ESCOBAR ESCOBAR \\
		\textbf{TUTOR:} M. SC. MARCELO PALMA SALAS \\
		LA PAZ – BOLIVIA \\
		Mayo, 2025
	\end{center}

	\newpage
	\thispagestyle{empty}
	\renewcommand{\contentsname}{ÍNDICE}
	\tableofcontents

	\newpage
	\section*{Resumen}
	\addcontentsline{toc}{section}{Resumen}
	Esta monografía aborda la integración de tecnologías emergentes, específicamente la inteligencia artificial generativa y la realidad aumentada... [resumen truncado para brevedad]

	\textbf{Palabras clave:} Tecnologías emergentes, inteligencia artificial generativa, realidad aumentada, competencias digitales, educación superior.


	\newpage
	\section*{Abstract}
	\addcontentsline{toc}{section}{Abstract}
	This monograph addresses the integration of emerging technologies, specifically generative artificial intelligence and augmented reality... [abstract truncado para brevedad]

	\textbf{Keywords:} Emerging technologies, generative artificial intelligence, augmented reality, digital skills, higher education.

	\newpage
	\chapter{Introducción}
	La educación, como pilar fundamental del desarrollo humano, ha experimentado transformaciones en las últimas décadas... [introducción truncada para brevedad]

	\chapter{Marco Teórico}
	\lipsum[1-2]
	Ejemplo de cita con autor en el texto: \textcite{perez2021educacion}.\\
	Ejemplo de cita entre paréntesis: \parencite{smith2020ai}.

	\chapter{Marco Aplicativo}
	\lipsum[1-2]

	\chapter{Conclusiones y Recomendaciones}
	\lipsum[1-2]

	\chapter{Bibliografía}
	\lipsum[1]
	\printbibliography

\end{document}
	